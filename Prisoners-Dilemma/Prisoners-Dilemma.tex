\documentclass[11pt,preprint]{elsarticle}

\usepackage{lmodern}
%%%% My spacing
\usepackage{setspace}
\setstretch{1.2}
\DeclareMathSizes{12}{14}{10}{10}

% Wrap around which gives all figures included the [H] command, or places it "here". This can be tedious to code in Rmarkdown.
\usepackage{float}
\let\origfigure\figure
\let\endorigfigure\endfigure
\renewenvironment{figure}[1][2] {
    \expandafter\origfigure\expandafter[H]
} {
    \endorigfigure
}

\let\origtable\table
\let\endorigtable\endtable
\renewenvironment{table}[1][2] {
    \expandafter\origtable\expandafter[H]
} {
    \endorigtable
}


\usepackage{ifxetex,ifluatex}
\usepackage{fixltx2e} % provides \textsubscript
\ifnum 0\ifxetex 1\fi\ifluatex 1\fi=0 % if pdftex
  \usepackage[T1]{fontenc}
  \usepackage[utf8]{inputenc}
\else % if luatex or xelatex
  \ifxetex
    \usepackage{mathspec}
    \usepackage{xltxtra,xunicode}
  \else
    \usepackage{fontspec}
  \fi
  \defaultfontfeatures{Mapping=tex-text,Scale=MatchLowercase}
  \newcommand{\euro}{€}
\fi

\usepackage{amssymb, amsmath, amsthm, amsfonts}

\def\bibsection{\section*{References}} %%% Make "References" appear before bibliography


\usepackage[numbers]{natbib}

\usepackage{longtable}
\usepackage[margin=2.3cm,bottom=2cm,top=2.5cm, includefoot]{geometry}
\usepackage{fancyhdr}
\usepackage[bottom, hang, flushmargin]{footmisc}
\usepackage{graphicx}
\numberwithin{equation}{section}
\numberwithin{figure}{section}
\numberwithin{table}{section}
\setlength{\parindent}{0cm}
\setlength{\parskip}{1.3ex plus 0.5ex minus 0.3ex}
\usepackage{textcomp}
\renewcommand{\headrulewidth}{0.2pt}
\renewcommand{\footrulewidth}{0.3pt}

\usepackage{array}
\newcolumntype{x}[1]{>{\centering\arraybackslash\hspace{0pt}}p{#1}}

%%%%  Remove the "preprint submitted to" part. Don't worry about this either, it just looks better without it:
\makeatletter
\def\ps@pprintTitle{%
  \let\@oddhead\@empty
  \let\@evenhead\@empty
  \let\@oddfoot\@empty
  \let\@evenfoot\@oddfoot
}
\makeatother

 \def\tightlist{} % This allows for subbullets!

\usepackage{hyperref}
\hypersetup{breaklinks=true,
            bookmarks=true,
            colorlinks=true,
            citecolor=blue,
            urlcolor=blue,
            linkcolor=blue,
            pdfborder={0 0 0}}


% The following packages allow huxtable to work:
\usepackage{siunitx}
\usepackage{multirow}
\usepackage{hhline}
\usepackage{calc}
\usepackage{tabularx}
\usepackage{booktabs}
\usepackage{caption}


\newenvironment{columns}[1][]{}{}

\newenvironment{column}[1]{\begin{minipage}{#1}\ignorespaces}{%
\end{minipage}
\ifhmode\unskip\fi
\aftergroup\useignorespacesandallpars}

\def\useignorespacesandallpars#1\ignorespaces\fi{%
#1\fi\ignorespacesandallpars}

\makeatletter
\def\ignorespacesandallpars{%
  \@ifnextchar\par
    {\expandafter\ignorespacesandallpars\@gobble}%
    {}%
}
\makeatother


% definitions for citeproc citations
\NewDocumentCommand\citeproctext{}{}
\NewDocumentCommand\citeproc{mm}{%
\begingroup\def\citeproctext{#2}\cite{#1}\endgroup}
\makeatletter
% allow citations to break across lines
\let\@cite@ofmt\@firstofone
% avoid brackets around text for \cite:
\def\@biblabel#1{}
\def\@cite#1#2{{#1\if@tempswa , #2\fi}}
\makeatother
\newlength{\cslhangindent}
\setlength{\cslhangindent}{1.5em}
\newlength{\csllabelwidth}
\setlength{\csllabelwidth}{3em}
\newenvironment{CSLReferences}[2] % #1 hanging-indent, #2 entry-spacing
{\begin{list}{}{%
	\setlength{\itemindent}{0pt}
	\setlength{\leftmargin}{0pt}
	\setlength{\parsep}{0pt}
	% turn on hanging indent if param 1 is 1
	\ifodd #1
	\setlength{\leftmargin}{\cslhangindent}
	\setlength{\itemindent}{-1\cslhangindent}
	\fi
	% set entry spacing
	\setlength{\itemsep}{#2\baselineskip}}}
{\end{list}}

\usepackage{calc}
\newcommand{\CSLBlock}[1]{\hfill\break\parbox[t]{\linewidth}{\strut\ignorespaces#1\strut}}
\newcommand{\CSLLeftMargin}[1]{\parbox[t]{\csllabelwidth}{\strut#1\strut}}
\newcommand{\CSLRightInline}[1]{\parbox[t]{\linewidth - \csllabelwidth}{\strut#1\strut}}
\newcommand{\CSLIndent}[1]{\hspace{\cslhangindent}#1}


\urlstyle{same}  % don't use monospace font for urls
\setlength{\parindent}{0pt}
\setlength{\parskip}{6pt plus 2pt minus 1pt}
\setlength{\emergencystretch}{3em}  % prevent overfull lines
\setcounter{secnumdepth}{5}

%%% Use protect on footnotes to avoid problems with footnotes in titles
\let\rmarkdownfootnote\footnote%
\def\footnote{\protect\rmarkdownfootnote}
\IfFileExists{upquote.sty}{\usepackage{upquote}}{}

%%% Include extra packages specified by user

%%% Hard setting column skips for reports - this ensures greater consistency and control over the length settings in the document.
%% page layout
%% paragraphs
\setlength{\baselineskip}{12pt plus 0pt minus 0pt}
\setlength{\parskip}{12pt plus 0pt minus 0pt}
\setlength{\parindent}{0pt plus 0pt minus 0pt}
%% floats
\setlength{\floatsep}{12pt plus 0 pt minus 0pt}
\setlength{\textfloatsep}{20pt plus 0pt minus 0pt}
\setlength{\intextsep}{14pt plus 0pt minus 0pt}
\setlength{\dbltextfloatsep}{20pt plus 0pt minus 0pt}
\setlength{\dblfloatsep}{14pt plus 0pt minus 0pt}
%% maths
\setlength{\abovedisplayskip}{12pt plus 0pt minus 0pt}
\setlength{\belowdisplayskip}{12pt plus 0pt minus 0pt}
%% lists
\setlength{\topsep}{10pt plus 0pt minus 0pt}
\setlength{\partopsep}{3pt plus 0pt minus 0pt}
\setlength{\itemsep}{5pt plus 0pt minus 0pt}
\setlength{\labelsep}{8mm plus 0mm minus 0mm}
\setlength{\parsep}{\the\parskip}
\setlength{\listparindent}{\the\parindent}
%% verbatim
\setlength{\fboxsep}{5pt plus 0pt minus 0pt}



\begin{document}



\begin{frontmatter}  %

\title{Beyond Tit-for-Tat}

% Set to FALSE if wanting to remove title (for submission)




\author[Add1]{Liam Andrew Beattie}
\ead{22562435@sun.ac.za}

\author[Add1]{Abdul Qaadir Cassiem}
\ead{20863667@sun.ac.za}




\address[Add1]{Microeconomics 871, Stellenbosch University, South
Africa}


\begin{abstract}
\small{
Beyond Tit-for-tat: Set up a repeated prisoner's dilemma computer
tournament, in which strategies compete against each other. Write a
report on your findings.
}
\end{abstract}

\vspace{1cm}





\vspace{0.5cm}

\end{frontmatter}

\setcounter{footnote}{0}



%________________________
% Header and Footers
%%%%%%%%%%%%%%%%%%%%%%%%%%%%%%%%%
\pagestyle{fancy}
\chead{}
\rhead{}
\lfoot{}
\rfoot{\footnotesize Page \thepage}
\lhead{}
%\rfoot{\footnotesize Page \thepage } % "e.g. Page 2"
\cfoot{}

%\setlength\headheight{30pt}
%%%%%%%%%%%%%%%%%%%%%%%%%%%%%%%%%
%________________________

\headsep 35pt % So that header does not go over title




\section{\texorpdfstring{Introduction
\label{Introduction}}{Introduction }}\label{introduction}

Here are the articles that i have downloaded and are in the sources
folder. You might need to look at the bibref doc in ``/Tex'' to know
which abbreviated document name is which doc in the sources folder,
however the dates can help. Or you can knit the doc and then you should
be able to see. Lets compare and rate each source for its usefulness. I
still need to look over them and to select which one to incoporate in
the coding and essay.

This first paper is basically what we are trying to do (recreate)
Axelrod (\citeproc{ref-axelrod1980effective}{1980}). How can we improve
on it? New strategies maybe?

Lange \& Baylor (\citeproc{ref-lange2007teaching}{2007}), Farrell \&
Ware (\citeproc{ref-farrell1988evolutionary}{1989}), Kreps, Milgrom,
Roberts \& Wilson (\citeproc{ref-kreps1982rational}{1982}), Romero \&
Rosokha (\citeproc{ref-romero2018constructing}{2018}), Bó \& Fréchette
(\citeproc{ref-dalbo2019strategy}{2019}), Breitmoser
(\citeproc{ref-breitmoser2015cooperation}{2015}), Gaudesi, Piccolo,
Squillero \& Tonda (\citeproc{ref-gaudesi2016exploiting}{2016}), García
\& Veelen (\citeproc{ref-garcia2018no}{2018}), Embrey, Fréchette \&
Yuksel (\citeproc{ref-embrey2017cooperation}{2018}),

I guess when we also need to provide good notation and formatting for
the strategies we end up doing, lets be the neatest and the easiest to
understand.

I am also thinking on how the changing of the payoffs might change the
outcomes of our results, so to have two sections with different payoffs
might be beneficial.

\section{\texorpdfstring{Literature
Review\label{litreview}}{Literature Review}}\label{literature-review}

\section{Game Construction}\label{game-construction}

Ideally have a round robin type of tournament, the number of rounds
played still needs to be decided but one of the paper's i saw had 25
rounds. We need to also define how we aim to evaluate games and
strategies - typically we go by overall utility rather than amount of
games won. This is a utilitarian approach in the philosophical ethics
systems of the word.

Basing the results on utilitarian principles of maximum welfare makes us
really have to consider the pay-off values for our prisoners dilemma
game, possibly could play around with different values, as well as
include two payoff values in our results. (we need to consider how much
better is mutual cooperation than mutual defection).

More importantly we need to select which strategies to include, ideally
have the basic ones in there, and then create a few interesting ones
which we can explain and add more depth to the paper with the complex
ones:

Basic Strategies:

\begin{itemize}
\item
  Always Cooperate: This strategy always cooperates, regardless of the
  opponent's previous moves.
\item
  Always Defect: This strategy always defects, regardless of the
  opponent's previous moves.
\item
  Tit-for-Tat (TFT): Cooperates on the first move, then mimics the
  opponent's last move in subsequent rounds.
\item
  Grim Trigger: Cooperates until the opponent defects once, then defects
  forever.
\item
  Random: Randomly chooses to cooperate or defect with some probability.
\item
  Tit-for-Two-Tats: Similar to Tit-for-Tat but defects only after two
  consecutive defections by the available player.
\item
  Pavlov (Win-Stay, Lose-Shift): Cooperates if the last round was a
  success (mutual cooperation or mutual defection), otherwise defects.
\end{itemize}

More Complex Strategies:

\begin{itemize}
\item
  Generous Tit-for-Tat: Similar to Tit-for-Tat, but occasionally
  forgives a defection.
\item
  Tit-for-Tat with Randomization: A variant of Tit-for-Tat where the
  player may defect or cooperate with a certain probability after the
  opponent defects.
\item
  Tit-for-Tat with Forgiveness: Like TFT but occasionally forgives a
  defection, returning to cooperation.
\end{itemize}

Of course there are more strategies that can deal with memory length and
what not, and maybe change startegies based on the past behaviour of the
opponent.

\begin{verbatim}
##            Player 2
## Player 1    Cooperate Defect
##   Cooperate         3      0
##   Defect            5      1
\end{verbatim}

\section{Conclusion}\label{conclusion}

\newpage

\section*{References}\label{references}
\addcontentsline{toc}{section}{References}

\phantomsection\label{refs}
\begin{CSLReferences}{1}{1}
\bibitem[\citeproctext]{ref-axelrod1980effective}
Axelrod, R. 1980. Effective choice in the prisoner's dilemma. \emph{The
Journal of Conflict Resolution}. 24(1):3--25. {[}Online{]}, Available:
\url{http://www.jstor.org/stable/173932}.

\bibitem[\citeproctext]{ref-dalbo2019strategy}
Bó, P.D. \& Fréchette, G.R. 2019.
\href{https://doi.org/10.1257/aer.20181480}{Strategy choice in the
infinitely repeated prisoner's dilemma}. \emph{American Economic
Review}. 109(11):3929--3952.

\bibitem[\citeproctext]{ref-breitmoser2015cooperation}
Breitmoser, Y. 2015.
\href{https://doi.org/10.1257/aer.20130675}{Cooperation, but no
reciprocity: Individual strategies in the repeated prisoner's dilemma}.
\emph{American Economic Review}. 105(9):2882--2910.

\bibitem[\citeproctext]{ref-embrey2017cooperation}
Embrey, M., Fréchette, G.R. \& Yuksel, S. 2018.
\href{https://doi.org/10.1093/qje/qjx033}{Cooperation in the finitely
repeated prisoner's dilemma}. \emph{The Quarterly Journal of Economics}.
133(2):509--551.

\bibitem[\citeproctext]{ref-farrell1988evolutionary}
Farrell, J. \& Ware, R. 1989. Evolutionary stability in the repeated
prisoner's dilemma. \emph{Journal of Economic Theory}. 47(1):1--12.

\bibitem[\citeproctext]{ref-garcia2018no}
García, J. \& Veelen, M. van. 2018.
\href{https://doi.org/10.3389/frobt.2018.00102}{No strategy can win in
the repeated prisoner's dilemma: Linking game theory and computer
simulations}. \emph{Frontiers in Robotics and AI}. 5:102.

\bibitem[\citeproctext]{ref-gaudesi2016exploiting}
Gaudesi, M., Piccolo, E., Squillero, G. \& Tonda, A. 2016. Exploiting
evolutionary modeling to prevail in iterated prisoner's dilemma
tournaments. \emph{IEEE Transactions on Computational Intelligence and
AI in Games}. 8(3):235--247.

\bibitem[\citeproctext]{ref-kreps1982rational}
Kreps, D.M., Milgrom, P., Roberts, J. \& Wilson, R. 1982. Rational
cooperation in the finitely repeated prisoners' dilemma. \emph{Journal
of Economic Theory}. 27(2):245--252.

\bibitem[\citeproctext]{ref-lange2007teaching}
Lange, C. \& Baylor, A.L. 2007.
\href{https://doi.org/10.3200/JECE.38.4.407-418}{Teaching the repeated
prisoner's dilemma with a computerized tournament}. \emph{The Journal of
Economic Education}. 38(4):407--418.

\bibitem[\citeproctext]{ref-romero2018constructing}
Romero, J. \& Rosokha, Y. 2018.
\href{https://doi.org/10.1016/j.euroecorev.2018.02.008}{Constructing
strategies in the indefinitely repeated prisoner's dilemma game}.
\emph{European Economic Review}. 104:185--219.

\end{CSLReferences}

\bibliography{Tex/ref}





\end{document}
